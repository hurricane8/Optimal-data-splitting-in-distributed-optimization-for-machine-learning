\documentclass{article}

\usepackage{arxiv}

\usepackage[utf8]{inputenc} % allow utf-8 input
\usepackage[T1]{fontenc}    % use 8-bit T1 fonts
\usepackage{hyperref}       % hyperlinks
\usepackage{url}            % simple URL typesetting
\usepackage{booktabs}       % professional-quality tables
\usepackage{amsfonts}       % blackboard math symbols
\usepackage{nicefrac}       % compact symbols for 1/2, etc.
\usepackage{microtype}      % microtypography
\usepackage{lipsum}		% Can be removed after putting your text content
\usepackage{graphicx}
\usepackage{natbib}
\usepackage{doi}
\usepackage{amsmath}
\usepackage{amssymb}


\title{A template for the \emph{arxiv} style}

%\date{September 9, 1985}	% Here you can change the date presented in the paper title
%\date{} 					% Or removing it

\author{David S.~Hippocampus\thanks{Use footnote for providing further
		information about author (webpage, alternative
		address)} \\
	Department of Computer Science\\
	Cranberry-Lemon University\\
	Pittsburgh, PA 15213 \\
	\texttt{hippo@cs.cranberry-lemon.edu} \\
	%% examples of more authors
	\And
	Elias D.~Striatum \\
	Department of Electrical Engineering\\
	Mount-Sheikh University\\
	Santa Narimana, Levand \\
	\texttt{stariate@ee.mount-sheikh.edu} \\
	%% \AND
	%% Coauthor \\
	%% Affiliation \\
	%% Address \\
	%% \texttt{email} \\
	%% \And
	%% Coauthor \\
	%% Affiliation \\
	%% Address \\
	%% \texttt{email} \\
	%% \And
	%% Coauthor \\
	%% Affiliation \\
	%% Address \\
	%% \texttt{email} \\
}

% Uncomment to remove the date
%\date{}

% Uncomment to override  the `A preprint' in the header
%\renewcommand{\headeright}{Technical Report}
%\renewcommand{\undertitle}{Technical Report}
\renewcommand{\shorttitle}{\textit{arXiv} Template}

%%% Add PDF metadata to help others organize their library
%%% Once the PDF is generated, you can check the metadata with
%%% $ pdfinfo template.pdf
\hypersetup{
pdftitle={A template for the arxiv style},
pdfsubject={q-bio.NC, q-bio.QM},
pdfauthor={David S.~Hippocampus, Elias D.~Striatum},
pdfkeywords={First keyword, Second keyword, More},
}

\begin{document}
\maketitle

\begin{abstract}
	\lipsum[1]
\end{abstract}


% keywords can be removed
\keywords{First keyword \and Second keyword \and More}


\section{Introduction}
\lipsum[2]
\lipsum[3]


\section{Headings: first level}
\label{sec:headings}

\lipsum[4] See Section \ref{sec:headings}.

\subsection{Headings: second level}
\lipsum[5]
\begin{equation}
	\xi _{ij}(t)=P(x_{t}=i,x_{t+1}=j|y,v,w;\theta)= {\frac {\alpha _{i}(t)a^{w_t}_{ij}\beta _{j}(t+1)b^{v_{t+1}}_{j}(y_{t+1})}{\sum _{i=1}^{N} \sum _{j=1}^{N} \alpha _{i}(t)a^{w_t}_{ij}\beta _{j}(t+1)b^{v_{t+1}}_{j}(y_{t+1})}}
\end{equation}

\subsubsection{Headings: third level}
\lipsum[6]

\paragraph{Paragraph}
\lipsum[7]



\section{Examples of citations, figures, tables, references}
\label{sec:others}

\subsection{Citations}
\cite{nesterov2003introductory}
Citations use \verb+natbib+. The documentation may be found at
\begin{center}
	\url{http://mirrors.ctan.org/macros/latex/contrib/natbib/natnotes.pdf}
\end{center}

Here is an example usage of the two main commands (\verb+citet+ and \verb+citep+): Some people thought a thing \citep{kour2014real, hadash2018estimate} but other people thought something else \citep{kour2014fast}. Many people have speculated that if we knew exactly why \citet{kour2014fast} thought this\dots

\subsection{Figures}
\lipsum[10]
See Figure \ref{fig:fig1}. Here is how you add footnotes. \footnote{Sample of the first footnote.}
\lipsum[11]

\begin{figure}
	\centering
	\fbox{\rule[-.5cm]{4cm}{4cm} \rule[-.5cm]{4cm}{0cm}}
	\caption{Sample figure caption.}
	\label{fig:fig1}
\end{figure}

\subsection{Tables}
See awesome Table~\ref{tab:table}.

The documentation for \verb+booktabs+ (`Publication quality tables in LaTeX') is available from:
\begin{center}
	\url{https://www.ctan.org/pkg/booktabs}
\end{center}


\begin{table}
	\caption{Sample table title}
	\centering
	\begin{tabular}{lll}
		\toprule
		\multicolumn{2}{c}{Part}                   \\
		\cmidrule(r){1-2}
		Name     & Description     & Size ($\mu$m) \\
		\midrule
		Dendrite & Input terminal  & $\sim$100     \\
		Axon     & Output terminal & $\sim$10      \\
		Soma     & Cell body       & up to $10^6$  \\
		\bottomrule
	\end{tabular}
	\label{tab:table}
\end{table}

\subsection{Lists}
\begin{itemize}
	\item Lorem ipsum dolor sit amet
	\item consectetur adipiscing elit.
	\item Aliquam dignissim blandit est, in dictum tortor gravida eget. In ac rutrum magna.
\end{itemize}

\subsection{More maths}

Use equations for single line maths
\begin{equation}
    \label{eq:1}
    f(x^{k+1}) - f(x^*) 
    \leq
    f(x^k) - f(x^*) + \eta_k \langle\nabla f(x^k), s^{k} - x^k \rangle + \frac{L \eta_k^2}{2} \|s^{k} - x^k \|^2
\end{equation}
To mention it in the text \eqref{eq:1}. If you don't want to mention it, use 
\begin{equation*}
    f(x^{k+1}) - f(x^*) 
    \leq
    f(x^k) - f(x^*) + \eta_k \langle\nabla f(x^k), s^{k} - x^k \rangle + \frac{L \eta_k^2}{2} \|s^{k} - x^k \|^2
\end{equation*}
or 
\begin{equation}
    f(x^{k+1}) - f(x^*) 
    \leq
    f(x^k) - f(x^*) + \eta_k \langle\nabla f(x^k), s^{k} - x^k \rangle + \frac{L \eta_k^2}{2} \|s^{k} - x^k \|^2 \notag
\end{equation}
If you want to have multi-line maths:
\begin{eqnarray}
    f(x^{k+1}) - f(x^*) 
    &\leq&
    f(x^k) - f(x^*) + \eta_k \langle\nabla f(x^k), s^{k} - x^k \rangle + \frac{L \eta_k^2}{2} \|s^{k} - x^k \|^2
    \\&=&
    f(x^k) - f(x^*) + \eta_k \langle\nabla f(x^k), s^{k} - x^k \rangle + \frac{L \eta_k^2}{2} \|s^{k} - x^k \|^2
\end{eqnarray}
Here we see that have two tags. Fix it
\begin{eqnarray}
    f(x^{k+1}) - f(x^*) 
    &\leq&
    f(x^k) - f(x^*) + \eta_k \langle\nabla f(x^k), s^{k} - x^k \rangle + \frac{L \eta_k^2}{2} \|s^{k} - x^k \|^2
    \notag\\&=&
    f(x^k) - f(x^*) + \eta_k \langle\nabla f(x^k), s^{k} - x^k \rangle + \frac{L \eta_k^2}{2} \|s^{k} - x^k \|^2
\end{eqnarray}
or
\begin{eqnarray}
    \begin{split}
        f(x^{k+1}) - f(x^*) 
        \leq&
        f(x^k) - f(x^*) + \eta_k \langle\nabla f(x^k), s^{k} - x^k \rangle + \frac{L \eta_k^2}{2} \|s^{k} - x^k \|^2
        \\=&
        f(x^k) - f(x^*) + \eta_k \langle\nabla f(x^k), s^{k} - x^k \rangle + \frac{L \eta_k^2}{2} \|s^{k} - x^k \|^2
    \end{split}
\end{eqnarray}
Instead of eqnarray, we can use
\begin{align*}
    f(x^{k+1}) - f(x^*) 
    \leq&
    f(x^k) - f(x^*) + \eta_k \langle\nabla f(x^k), s^{k} - x^k \rangle + \frac{L \eta_k^2}{2} \|s^{k} - x^k \|^2
    \\=&
    f(x^k) - f(x^*) + \eta_k \langle\nabla f(x^k), s^{k} - x^k \rangle + \frac{L \eta_k^2}{2} \|s^{k} - x^k \|^2
\end{align*}
Please, play with \& in eqnarray, split and align to understand how it is works.


\bibliographystyle{unsrtnat}
\bibliography{references}  %%% Uncomment this line and comment out the ``thebibliography'' section below to use the external .bib file (using bibtex) .


%%% Uncomment this section and comment out the \bibliography{references} line above to use inline references.
% \begin{thebibliography}{1}

% 	\bibitem{kour2014real}
% 	George Kour and Raid Saabne.
% 	\newblock Real-time segmentation of on-line handwritten arabic script.
% 	\newblock In {\em Frontiers in Handwriting Recognition (ICFHR), 2014 14th
% 			International Conference on}, pages 417--422. IEEE, 2014.

% 	\bibitem{kour2014fast}
% 	George Kour and Raid Saabne.
% 	\newblock Fast classification of handwritten on-line arabic characters.
% 	\newblock In {\em Soft Computing and Pattern Recognition (SoCPaR), 2014 6th
% 			International Conference of}, pages 312--318. IEEE, 2014.

% 	\bibitem{hadash2018estimate}
% 	Guy Hadash, Einat Kermany, Boaz Carmeli, Ofer Lavi, George Kour, and Alon
% 	Jacovi.
% 	\newblock Estimate and replace: A novel approach to integrating deep neural
% 	networks with existing applications.
% 	\newblock {\em arXiv preprint arXiv:1804.09028}, 2018.

% \end{thebibliography}


\end{document}
